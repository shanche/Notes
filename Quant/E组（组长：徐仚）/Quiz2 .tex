\documentclass[12pt]{article}
%%%%%%%%%%%%%%%%%%%%%%%%%%%%%%%%%%%%%%%%%%%%%%%%%%%%%%%%%%%%%%%%%%%%%%%%%%%%%%%%%%%%%%%%%%%%%%%%%%%%%%%%%%%%%%%%%%%%%%%%%%%%%%%%%%%%%%%%%%%%%%%%%%%%%%%%%%%%%%%%%%%%%%%%%%%%%%%%%%%%%%%%%%%%%%%%%%%%%%%%%%%%%%%%%%%%%%%%%%%%%%%%%%%%%%%%%%%%%%%%%%%%%%%%%%%%
\usepackage{amsfonts}
\usepackage{amsmath}
\usepackage{amssymb}
\usepackage{geometry}
\usepackage{indentfirst}
%\usepackage{booktabs} % Fancy tables
\usepackage{graphicx}
\usepackage{multirow} 
\usepackage{pdflscape}
\usepackage{rotating}
\usepackage{listings}
\usepackage{color}

\definecolor{dkgreen}{rgb}{0,0.6,0}
\definecolor{gray}{rgb}{0.5,0.5,0.5}
\definecolor{mauve}{rgb}{0.58,0,0.82}

\lstset{frame=tb,
	language=C++,
	aboveskip=3mm,
	belowskip=3mm,
	showstringspaces=false,
	columns=flexible,
	basicstyle={\small\ttfamily},
	numbers=none,
	numberstyle=\tiny\color{gray},
	keywordstyle=\color{blue},
	commentstyle=\color{dkgreen},
	stringstyle=\color{mauve},
	breaklines=true,
	breakatwhitespace=true,
	tabsize=3
}

\setcounter{MaxMatrixCols}{10}
%TCIDATA{OutputFilter=LATEX.DLL}
%TCIDATA{Version=5.50.0.2960}
%TCIDATA{<META NAME="SaveForMode" CONTENT="1">}
%TCIDATA{BibliographyScheme=Manual}
%TCIDATA{Created=Sunday, August 22, 2010 22:02:08}
%TCIDATA{LastRevised=Thursday, September 08, 2011 19:40:06}
%TCIDATA{<META NAME="GraphicsSave" CONTENT="32">}
%TCIDATA{<META NAME="DocumentShell" CONTENT="Standard LaTeX\Blank - Standard LaTeX Article">}
%TCIDATA{Language=American English}
%TCIDATA{CSTFile=40 LaTeX article.cst}

\geometry{left=.8in,right=.8in,top=.8in,bottom=.8in}
\setlength{\parskip}{3pt plus 1pt minus 1pt}
\setlength{\parindent}{1.5em}
\newtheorem{theorem}{Theorem}
\newtheorem{acknowledgement}[theorem]{Acknowledgement}
\newtheorem{algorithm}[theorem]{Algorithm}
\newtheorem{axiom}[theorem]{Axiom}
\newtheorem{case}[theorem]{Case}
\newtheorem{claim}[theorem]{Claim}
\newtheorem{conclusion}[theorem]{Conclusion}
\newtheorem{condition}[theorem]{Condition}
\newtheorem{conjecture}[theorem]{Conjecture}
\newtheorem{corollary}[theorem]{Corollary}
\newtheorem{criterion}[theorem]{Criterion}
\newtheorem{definition}[theorem]{Definition}
\newtheorem{example}[theorem]{Example}
\newtheorem{exercise}[theorem]{Exercise}
\newtheorem{lemma}[theorem]{Lemma}
\newtheorem{notation}[theorem]{Notation}
\newtheorem{problem}[theorem]{Problem}
\newtheorem{proposition}[theorem]{Proposition}
\newtheorem{remark}[theorem]{Remark}
\newtheorem{solution}[theorem]{Solution}
\newtheorem{summary}[theorem]{Summary}
\newenvironment{proof}[1][Proof]{\noindent\textbf{#1.} }{\ \rule{0.5em}{0.5em}}
%\input{tcilatex}
\begin{document}

\title{Quiz 2}
\maketitle
%\maketitle
\section*{Math/Stat}

2) Given a 2 by 3 grid (which has 6 blocks and 17 edges), shortest route to visit all edges (assuming edge length is 1).\begin{solution}Recall a well known fact that one can visit each edge of a (connected) graph $G$ {\bf exactly once} if and only if the number of vertices of odd degree is either $0$ or $2$ (the degree of a vertex $v$ of $G$ is the number of edges passing through that vertex in question). Our graph $G$ in question has $6$ vertices of degree $3$, namely $T_1$, $T_2$ on the top side, $B_1$, $B_2$ one the bottom side, $L$ on the left side and $R$ on the right side.  Thus there is no way to find a path to visit all edges exactly once. However, if we make two auxiliary edges connecting $T_1$, $T_2$ and $B_1$, $B_2$, then there are only two vertices of degree $3$ left, namely $L$ and $R$. One can find a path to visiting each edge of the new graph $G'$ exactly once. This path has length 19 since the new graph has $19$ edges. This means on the original graph $G$, there exists a path of length $19$ to visit all edges (the path will visit edges $T_1T_2$ and $B_1B_2$ twice and all other edges once). It is easy to show this is the shortest path possible.

\end{solution}

5) Given a stick, if randomly cut into 3 pieces, what's the average size of the smallest, of the middle-sized, and of the largest pieces?
\begin{solution}Let $X$ and $Y$ be the random variable representing two of pieces, the thrid piece has length $1-X-Y$. Thus $X$ and $Y$ are i.i.d with $\mathcal{U}(0,1)$ distribution. We compute the c.d.f $F_{\max}$ of $\max{X,Y,1-X-Y}$ as follows:
\begin{eqnarray*}F_{\max}(t):&=&\mathbb{P}(\max\{X,Y,1-X-Y\}\le t\ |\ X+Y\le1)\\
&=&\mathbb{P}(X\le t,\ Y\le t,\ 1-X-Y\le t\ |\  X+Y\le1)\\
&=&\frac{\mathbb{P}(X\le t, Y\le t, 1-t\le X+Y\le 1)}{\mathbb{P}(X+Y\le 1)}
\end{eqnarray*}
We easily compute that 
 $$F_{\max}(t)=\begin{cases}0& t\le \frac{1}{3},\\ (3t-1)^2 & \frac{1}{3}<t\le \frac{1}{2},\\ {1}-{3}(1-t)^2 &\frac{1}{2}<t\le1.
\end{cases}
$$
We further compute the density function and then the expectation $$\mathbb{E}_{\max}=\frac{11}{18}.$$
One can similarly compute the c.m.f of $\min\{X,Y,1-X-Y\}$:
$$F_{\min}(t)=\begin{cases}1-(1-3t)^2 & 0\le t\le \frac{1}{3}\\
1 & t>\frac{1}{3}.
\end{cases}
$$
The expectation is $\mathbb{E}_{\min}=\frac{1}{9}$. The middle piece therefore has expectation $\frac{5}{18}$.

\end{solution}


\section*{C++ Coding Question}

12) Implement the interface for matrix class in C++.

\begin{lstlisting}
#include <iostream>
#include<vector>
using namespace std;

template<typename T>

class Matrix
{
public:
    Matrix(int A, int B, T t):RowNumber(A),ColNumber(B) //constructor innitialize A*B matrix with entries t.
    {
        MyMatrix.resize(A);
        for(int i=0;i<A;i++)
            MyMatrix[i].resize(B,t);
        
        
    }
    Matrix(const Matrix& OneMatrix):RowNumber(OneMatrix.RowNumber),ColNumber(OneMatrix.ColNumber) //copy constructor
   {
       MyMatrix.resize(RowNumber);
    for(int i=0;i<RowNumber;i++)
            MyMatrix[i].resize(ColNumber);
        for(int i;i<RowNumber;i++)
            for(int j=0;j<ColNumber;j++)
                MyMatrix[i][j]=OneMatrix.MyMatrix[i][j];
    }
    
    Matrix& operator=(const Matrix& OneMatrix) //overloading assignment operator
    {
        if(RowNumber!=OneMatrix.RowNumber||ColNumber!=OneMatrix.ColNumber)
        {cout<<"The matrices do not match."<<endl;
            throw -1;
        }
        for(int i=0;i<RowNumber;i++)
            for(int j=0;j<ColNumber;j++)
                MyMatrix[i][j]=OneMatrix.MyMatrix[i][j];
        return *this;
        

    }
    
    Matrix operator+(const Matrix& AnotherMatrix) //overloading +
    {
        if(RowNumber!=AnotherMatrix.RowNumber||ColNumber!=AnotherMatrix.ColNumber)
        {cout<<"The matrices do not match."<<endl;
            throw -1 ;
        }
        else
        {Matrix A(RowNumber,ColNumber,0);
            for(int i=0;i<RowNumber;i++)
                for(int j=0;j<ColNumber;j++)
                    A.MyMatrix[i][j]=MyMatrix[i][j]+AnotherMatrix.MyMatrix[i][j];
            return A;
        }
        
    }
    
    Matrix operator-(const Matrix& AnotherMatrix) //overloading -
    {
        if(RowNumber!=AnotherMatrix.RowNumber||ColNumber!=AnotherMatrix.ColNumber)
        {cout<<"The matrices do not match."<<endl;
            throw -1 ;
        }
        else
        {Matrix A(RowNumber,ColNumber,0);
            for(int i=0;i<RowNumber;i++)
            { for(int j=0;j<ColNumber;j++)
                    A.MyMatrix[i][j]=MyMatrix[i][j]-AnotherMatrix.MyMatrix[i][j];
            }
            return A;
        }

    }
    Matrix operator*(const Matrix& AnotherMatrix) //overloading matrix multiplication
    {
        if(ColNumber!=AnotherMatrix.RowNumber)
        {throw "The matrices can not be multiplied";
            
        }
        Matrix A(RowNumber,AnotherMatrix.ColNumber,0);
        for(int i=0;i<RowNumber;i++)
            for(int j=0;j<AnotherMatrix.ColNumber;j++)
            { for(int k=0;k<ColNumber;k++)
                    A.MyMatrix[i][j]+=MyMatrix[i][k]*AnotherMatrix.MyMatrix[k][j];
            }
        return A;
        
    }
    
    T& operator()(int A, int B)
    {   if(A>=RowNumber||B>=ColNumber||A<0||B<0)
    
        throw "Position not available";
        
    
        return MyMatrix[A][B];
    }
    
    void print()
    {
        for(int i=0;i<RowNumber;i++)
        { cout<<"\n";
            for(int j=0;j<ColNumber;j++)
                cout<<MyMatrix[i][j]<<'\t';}
        cout<<'\n';
    }
private:
    int RowNumber;
    int ColNumber;
    vector<vector<T>> MyMatrix;
};


\end{lstlisting}


\end{document}
