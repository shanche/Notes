%
\documentclass[12pt]{amsart}
%\usepackage{txfonts}      %{article} was 12pt latex e
\usepackage{amssymb}
\usepackage{eucal}
\usepackage{amsmath}
\usepackage{amscd}
\usepackage[dvips]{color}
\usepackage{multicol}
\usepackage[all]{xy}           %xypic macro for latex2.09
\usepackage{graphicx}
\usepackage{color}
\usepackage{colordvi}
\usepackage{xspace}
\usepackage{listings}
\usepackage{framed}
\usepackage{lipsum}
%\usepackage{axodraw}
\usepackage{hyperref}

\usepackage[active]{srcltx} %SRC Specials for DVI Searching


\renewcommand\baselinestretch{1}    %was    1, 1.5 for double sp

%%standard setting
%\topmargin -0.3truein \textheight 8.4truein \oddsidemargin 0.2truein
%\evensidemargin 0.2truein \textwidth 440pt
%=========================================================================================
%%little larger standard setting: good setting
\topmargin -.8cm \textheight 22.8cm \oddsidemargin 0cm
\evensidemargin -0cm \textwidth 16.3cm
%==================================================================================
%% facing large setting
%\topmargin -.8cm \textheight 22.8cm \oddsidemargin -2cm
%\evensidemargin 2cm \textwidth 15cm
%==================================================================================
%%wide %% small font, fit window
%\topmargin -3.3cm \textheight 27.5cm \oddsidemargin -2cm
%\evensidemargin -2cm \textwidth 20cm
%%%%%%%%%%%%%%%================================================
%note setting, fit window
%%wide note setting, fit window
%\topmargin -1.6cm \textheight 23cm \oddsidemargin -0.9cm
%\textwidth 33cm \evensidemargin -0.9cm
%==================================================================================
%%wide note setting, no margin
%\topmargin -1.6cm \textheight 14.25cm \oddsidemargin -2cm
%\textwidth 20cm \evensidemargin -1.9cm
%=================================================================================%
%print narrow note setting
%\topmargin -0.5truein \textheight 9.8truein \oddsidemargin -0.7truein \evensidemargin -0.7truein \textwidth 340pt
%=================================================================================
%\makeatletter

%\makeatletter

\begin{document}  %for latex 2.09
%\input amssym.def %
%\input amssym      %

\newcommand{\nc}{\newcommand}
\newcommand{\delete}[1]{}
\nc{\dfootnote}[1]{{}}          %{{}}
\nc{\ffootnote}[1]{\dfootnote{#1}}
%\nc{\mfootnote}[1]{{}}        % Use this to suppress footnotes
\nc{\mfootnote}[1]{\footnote{#1}} % Use this to show footnotes
\nc{\todo}[1]{\tred{To do:} #1}

%\delete{
\nc{\mlabel}[1]{\label{#1}}  % Use this to suppress names
\nc{\mcite}[1]{\cite{#1}}  % Use this to suppress names
\nc{\mref}[1]{\ref{#1}}  % Use this to suppress names
\nc{\mbibitem}[1]{\bibitem{#1}} % Use this to show number
%}


\delete{
\nc{\mlabel}[1]{\label{#1}  % Use the next two lines to show names
{\hfill \hspace{1cm}{\bf{{\ }\hfill(#1)}}}}
\nc{\mcite}[1]{\cite{#1}{{\bf{{\ }(#1)}}}}  % Use this lines to show names
\nc{\mref}[1]{\ref{#1}{{\bf{{\ }(#1)}}}}  % Use this lines to show names
\nc{\mbibitem}[1]{\bibitem[\bf #1]{#1}} % Use this to show name
}

\nc{\mkeep}[1]{\marginpar{{\bf #1}}} % Use this to show marginpar
%\nc{\mkeep}[1]{{}}      % Use this to suppress marginpar

%%%%%%%%%%%%%%%%%%%%%%%% Statements
\newtheorem{theorem}{Theorem}[section]
%\newtheorem{prop}{Proposition}[section]
\newtheorem{prop}[theorem]{Proposition}
%\newtheorem{defn}{Definition}[section]
\newtheorem{defn}[theorem]{Definition}
\newtheorem{lemma}[theorem]{Lemma}
\newtheorem{coro}[theorem]{Corollary}
\newtheorem{prop-def}[theorem]{Proposition-Definition}
\newtheorem{claim}{Claim}[section]
\newtheorem{remark}[theorem]{Remark}
\newtheorem{propprop}{Proposed Proposition}[section]
\newtheorem{conjecture}{Conjecture}
\newtheorem{exam}[theorem]{Example}
\newtheorem{assumption}{Assumption}
\newtheorem{condition}[theorem]{Assumption}

\renewcommand{\labelenumi}{{\rm(\roman{enumi})}}
\renewcommand{\theenumi}{\roman{enumi}}

\newcommand{\vectornorm}[1]{\left|\left|#1\right|\right|}

\nc{\tred}[1]{\textcolor{red}{#1}}
\nc{\tblue}[1]{\textcolor{blue}{#1}}
\nc{\tgreen}[1]{\textcolor{green}{#1}}
\nc{\tpurple}[1]{\textcolor{purple}{#1}}
\nc{\btred}[1]{\textcolor{red}{\bf #1}}
\nc{\btblue}[1]{\textcolor{blue}{\bf #1}}
\nc{\btgreen}[1]{\textcolor{green}{\bf #1}}
\nc{\btpurple}[1]{\textcolor{purple}{\bf #1}}

\nc{\li}[1]{\textcolor{red}{Li:#1}}
\nc{\cm}[1]{\textcolor{blue}{Chengming: #1}}
\nc{\xiang}[1]{\textcolor{green}{Xiang: #1}}


%%%%%%%%%%%%%%%%%%%%%%% symbols
\nc{\adec}{\check{;}} \nc{\aop}{\alpha}
\nc{\dftimes}{\widetilde{\otimes}} \nc{\dfl}{\succ}
\nc{\dfr}{\prec} \nc{\dfc}{\circ} \nc{\dfb}{\bullet}
\nc{\dft}{\star} \nc{\dfcf}{{\mathbf k}} \nc{\spr}{\cdot}
\nc{\twopr}{\circ} \nc{\tspr}{\star} \nc{\sempr}{\ast}
\nc{\disp}[1]{\displaystyle{#1}}
\nc{\bin}[2]{ (_{\stackrel{\scs{#1}}{\scs{#2}}})}  %binomial coeff
\nc{\binc}[2]{ \left (\!\! \begin{array}{c} \scs{#1}\\
    \scs{#2} \end{array}\!\! \right )}  %binomial coeff
\nc{\bincc}[2]{  \left ( {\scs{#1} \atop
    \vspace{-.5cm}\scs{#2}} \right )}  %binomial coeff
\nc{\sarray}[2]{\begin{array}{c}#1 \vspace{.1cm}\\ \hline
    \vspace{-.35cm} \\ #2 \end{array}}
\nc{\bs}{\bar{S}} \nc{\dcup}{\stackrel{\bullet}{\cup}}
\nc{\dbigcup}{\stackrel{\bullet}{\bigcup}} \nc{\etree}{\big |}
\nc{\la}{\longrightarrow} \nc{\fe}{\'{e}} \nc{\rar}{\rightarrow}
\nc{\dar}{\downarrow} \nc{\dap}[1]{\downarrow
\rlap{$\scriptstyle{#1}$}} \nc{\uap}[1]{\uparrow
\rlap{$\scriptstyle{#1}$}} \nc{\defeq}{\stackrel{\rm def}{=}}
\nc{\dis}[1]{\displaystyle{#1}} \nc{\dotcup}{\,
\displaystyle{\bigcup^\bullet}\ } \nc{\sdotcup}{\tiny{
\displaystyle{\bigcup^\bullet}\ }} \nc{\hcm}{\ \hat{,}\ }
\nc{\hcirc}{\hat{\circ}} \nc{\hts}{\hat{\shpr}}
\nc{\lts}{\stackrel{\leftarrow}{\shpr}}
\nc{\rts}{\stackrel{\rightarrow}{\shpr}} \nc{\lleft}{[}
\nc{\lright}{]} \nc{\uni}[1]{\tilde{#1}} \nc{\wor}[1]{\check{#1}}
\nc{\free}[1]{\bar{#1}} \nc{\den}[1]{\check{#1}} \nc{\lrpa}{\wr}
\nc{\curlyl}{\left \{ \begin{array}{c} {} \\ {} \end{array}
    \right .  \!\!\!\!\!\!\!}
\nc{\curlyr}{ \!\!\!\!\!\!\!
    \left . \begin{array}{c} {} \\ {} \end{array}
    \right \} }
\nc{\leaf}{\ell}       % number of leafs
\nc{\longmid}{\left | \begin{array}{c} {} \\ {} \end{array}
    \right . \!\!\!\!\!\!\!}
\nc{\ot}{\otimes} \nc{\sot}{{\scriptstyle{\ot}}}
\nc{\otm}{\overline{\ot}} \nc{\ora}[1]{\stackrel{#1}{\rar}}
\nc{\ola}[1]{\stackrel{#1}{\la}}%${\Bbb Z}$
\nc{\pltree}{\calt^\pl} \nc{\epltree}{\calt^{\pl,\NC}}
\nc{\rbpltree}{\calt^r} \nc{\scs}[1]{\scriptstyle{#1}}
\nc{\mrm}[1]{{\rm #1}}
\nc{\dirlim}{\displaystyle{\lim_{\longrightarrow}}\,}
\nc{\invlim}{\displaystyle{\lim_{\longleftarrow}}\,}
\nc{\mvp}{\vspace{0.5cm}} \nc{\svp}{\vspace{2cm}}
\nc{\vp}{\vspace{8cm}} \nc{\proofbegin}{\noindent{\bf Proof: }}
%\nc{\proofbegin}{\begin{proof}} % AMS command
\nc{\proofend}{$\blacksquare$ \vspace{0.5cm}}
%\nc{\proofend}{\end{proof}} %AMS command
\nc{\freerbpl}{{F^{\mathrm RBPL}}}
\nc{\sha}{{\mbox{\cyr X}}}  %used to be \cyr
\nc{\ncsha}{{\mbox{\cyr X}^{\mathrm NC}}} \nc{\ncshao}{{\mbox{\cyr
X}^{\mathrm NC,\,0}}} \nc{\rpr}{\circ} \nc{\apr}{\cdot}
\nc{\dpr}{{\tiny\diamond}} \nc{\rprpm}{{\rpr}}
\nc{\shpr}{\diamond}    %Shuffle product
\nc{\shprm}{\overline{\diamond}}    %Shuffle product
\nc{\shpro}{\diamond^0}    %Shuffle product
\nc{\shprr}{\diamond^r}     %product on controlled trees
\nc{\shpra}{\overline{\diamond}^r} \nc{\shpru}{\check{\diamond}}
\nc{\catpr}{\diamond_l} \nc{\rcatpr}{\diamond_r}
\nc{\lapr}{\diamond_a} \nc{\sqcupm}{\ot} \nc{\lepr}{\diamond_e}
\nc{\vep}{\varepsilon} \nc{\labs}{\mid\!} \nc{\rabs}{\!\mid}
\nc{\hsha}{\widehat{\sha}} \nc{\lsha}{\stackrel{\leftarrow}{\sha}}
\nc{\rsha}{\stackrel{\rightarrow}{\sha}} \nc{\lc}{\lfloor}
\nc{\rc}{\rfloor} \nc{\tpr}{\sqcup} \nc{\nctpr}{\vee}
\nc{\plpr}{\star} \nc{\rbplpr}{\bar{\plpr}} \nc{\sqmon}[1]{\langle
#1\rangle} \nc{\forest}{\calf} \nc{\ass}[1]{\alpha({#1})}
\nc{\altx}{\Lambda_X} \nc{\vecT}{\vec{T}} \nc{\onetree}{\bullet}
\nc{\Ao}{\check{A}} \nc{\seta}{\underline{\Ao}}
\nc{\deltaa}{\overline{\delta}} \nc{\trho}{\tilde{\rho}}


%%%%%%%%%%%%%%%%%%%%% roman fonts, in alphabetic order
\nc{\mmbox}[1]{\mbox{\ #1\ }} \nc{\ann}{\mrm{ann}}
\nc{\Aut}{\mrm{Aut}} \nc{\can}{\mrm{can}} \nc{\twoalg}{{two-sided
algebra}\xspace} \nc{\bwt}{{mass}\xspace}
\nc{\bop}{{modification}\xspace} \nc{\ewt}{{mass}\xspace}
\nc{\ewts}{{masses}\xspace} \nc{\tto}{{extended}\xspace}
\nc{\Tto}{{Extended}\xspace} \nc{\tte}{{extended}\xspace}
\nc{\gyb}{{generalized}\xspace} \nc{\Gyb}{{Generalized}\xspace}
\nc{\MAYBE}{{EAYBE}\xspace} \nc{\GAYBE}{{GAYBE}\xspace}
\nc{\esym}{\vep} \nc{\colim}{\mrm{colim}} \nc{\Cont}{\mrm{Cont}}
\nc{\rchar}{\mrm{char}} \nc{\cok}{\mrm{coker}} \nc{\dtf}{{R-{\rm
tf}}} \nc{\dtor}{{R-{\rm tor}}}
\renewcommand{\det}{\mrm{det}}
\nc{\depth}{{\mrm d}} \nc{\Div}{{\mrm Div}} \nc{\End}{\mrm{End}}
\nc{\Ext}{\mrm{Ext}} \nc{\Fil}{\mrm{Fil}} \nc{\Frob}{\mrm{Frob}}
\nc{\Gal}{\mrm{Gal}} \nc{\GL}{\mrm{GL}} \nc{\Hom}{\mrm{Hom}}
\nc{\hsr}{\mrm{H}} \nc{\hpol}{\mrm{HP}} \nc{\id}{\mrm{id}}
\nc{\im}{\mrm{im}} \nc{\incl}{\mrm{incl}}
\nc{\length}{\mrm{length}} \nc{\LR}{\mrm{LR}} \nc{\mchar}{\rm
char} \nc{\NC}{\mrm{NC}} \nc{\mpart}{\mrm{part}}
\nc{\pl}{\mrm{PL}} \nc{\ql}{{\QQ_\ell}} \nc{\qp}{{\QQ_p}}
\nc{\rank}{\mrm{rank}} \nc{\rba}{\rm{RBA }} \nc{\rbas}{\rm{RBAs }}
\nc{\rbpl}{\mrm{RBPL}} \nc{\rbw}{\rm{RBW }} \nc{\rbws}{\rm{RBWs }}
\nc{\rcot}{\mrm{cot}} \nc{\rest}{\rm{controlled}\xspace}
\nc{\rdef}{\mrm{def}} \nc{\rdiv}{{\rm div}} \nc{\rtf}{{\rm tf}}
\nc{\rtor}{{\rm tor}} \nc{\res}{\mrm{res}} \nc{\SL}{\mrm{SL}}
\nc{\Spec}{\mrm{Spec}} \nc{\tor}{\mrm{tor}} \nc{\Tr}{\mrm{Tr}}
\nc{\mtr}{\mrm{sk}} \nc{\type}{{\bwt}\xspace}

%%%%%%%%%%%%%%%%%% bold face
\nc{\ab}{\mathbf{Ab}} \nc{\Alg}{\mathbf{Alg}}
\nc{\Algo}{\mathbf{Alg}^0} \nc{\Bax}{\mathbf{Bax}}
\nc{\Baxo}{\mathbf{Bax}^0} \nc{\RB}{\mathbf{RB}}
\nc{\RBo}{\mathbf{RB}^0} \nc{\BRB}{\mathbf{RB}}
\nc{\Dend}{\mathbf{DD}} \nc{\bfk}{{\bf k}} \nc{\bfone}{{\bf 1}}
\nc{\base}[1]{{a_{#1}}} \nc{\detail}{\marginpar{\bf More detail}
    \noindent{\bf Need more detail!}
    \svp}
\nc{\Diff}{\mathbf{Diff}} \nc{\gap}{\marginpar{\bf
Incomplete}\noindent{\bf Incomplete!!}
    \svp}
\nc{\FMod}{\mathbf{FMod}} \nc{\mset}{\mathbf{MSet}}
\nc{\rb}{\mathrm{RB}} \nc{\Int}{\mathbf{Int}}
\nc{\Mon}{\mathbf{Mon}}
%\nc{\remark}{\noindent{\bf Remark: }}
\nc{\remarks}{\noindent{\bf Remarks: }}
\nc{\OS}{\mathbf{OS}} %free operated semigroup
\nc{\Rep}{\mathbf{Rep}} \nc{\Rings}{\mathbf{Rings}}
\nc{\Sets}{\mathbf{Sets}} \nc{\DT}{\mathbf{DT}}

%%%%%%%%%%%%%%%%%%%Bbb fonts
\nc{\BA}{{\mathbb A}} \nc{\CC}{{\mathbb C}} \nc{\DD}{{\mathbb D}}
\nc{\EE}{{\mathbb E}} \nc{\FF}{{\mathbb F}} \nc{\GG}{{\mathbb G}}
\nc{\HH}{{\mathbb H}} \nc{\LL}{{\mathbb L}} \nc{\NN}{{\mathbb N}}
\nc{\QQ}{{\mathbb Q}} \nc{\RR}{{\mathbb R}} \nc{\TT}{{\mathbb T}}
\nc{\VV}{{\mathbb V}} \nc{\ZZ}{{\mathbb Z}}


%%%%%%%%%%%%%%%%%%% cal fonts

\nc{\calao}{{\mathcal A}} \nc{\cala}{{\mathcal A}}
\nc{\calc}{{\mathcal C}} \nc{\cald}{{\mathcal D}}
\nc{\cale}{{\mathcal E}} \nc{\calf}{{\mathcal F}}
\nc{\calfr}{{{\mathcal F}^{\,r}}} \nc{\calfo}{{\mathcal F}^0}
\nc{\calfro}{{\mathcal F}^{\,r,0}} \nc{\oF}{\overline{F}}
\nc{\calg}{{\mathcal G}} \nc{\calh}{{\mathcal H}}
\nc{\cali}{{\mathcal I}} \nc{\calj}{{\mathcal J}}
\nc{\call}{{\mathcal L}} \nc{\calm}{{\mathcal M}}
\nc{\caln}{{\mathcal N}} \nc{\calo}{{\mathcal O}}
\nc{\calp}{{\mathcal P}} \nc{\calr}{{\mathcal R}}
\nc{\calt}{{\mathcal T}} \nc{\caltr}{{\mathcal T}^{\,r}}
\nc{\calu}{{\mathcal U}} \nc{\calv}{{\mathcal V}}
\nc{\calw}{{\mathcal W}} \nc{\calx}{{\mathcal X}}
\nc{\CA}{\mathcal{A}}


%%%%%%%%%%%%%%%%%%  frak fonts
\nc{\fraka}{{\mathfrak a}} \nc{\frakB}{{\mathfrak B}}
\nc{\frakb}{{\mathfrak b}} \nc{\frakd}{{\mathfrak d}}
\nc{\oD}{\overline{D}} \nc{\frakF}{{\mathfrak F}}
\nc{\frakg}{{\mathfrak g}} \nc{\frakk}{{\mathfrak k}}
\nc{\frakm}{{\mathfrak m}} \nc{\frakM}{{\mathfrak M}}
\nc{\frakMo}{{\mathfrak M}^0} \nc{\frakp}{{\mathfrak p}}
\nc{\frakS}{{\mathfrak S}} \nc{\frakSo}{{\mathfrak S}^0}
\nc{\fraks}{{\mathfrak s}} \nc{\os}{\overline{\fraks}}
\nc{\frakT}{{\mathfrak T}} \nc{\oT}{\overline{T}}
%\nc{\frakx}{{\mathfrak x}}
\nc{\frakX}{{\mathfrak X}} \nc{\frakXo}{{\mathfrak X}^0}
\nc{\frakx}{{\mathbf x}}
%\nc{\frakTxo}{{\frakTx}^0}
\nc{\frakTx}{\frakT}      %All rooted trees, correspond to \ncsha(X)
\nc{\frakTa}{\frakT^a}        % rooted trees for \ncsha(A)
\nc{\frakTxo}{\frakTx^0}   % rooted trees for \ncshao(X)
\nc{\caltao}{\calt^{a,0}}   % rooted trees for \ncshao(A)
\nc{\ox}{\overline{\frakx}} \nc{\fraky}{{\mathfrak y}}
\nc{\frakz}{{\mathfrak z}} \nc{\oX}{\overline{X}}

\font\cyr=wncyr10

\nc{\redtext}[1]{\textcolor{red}{#1}}


%%%%%%%%%%%%%%%%%%%%%%%%%%%%%%%%%%%%%%%%%%%%%%%%%%%%%%%%%%%%%%%%%%




\hfil {\bf \Huge{Quiz 3}}\hfil





\bigskip

\hfil {Group 2} \hfil




\bigskip

\section{Problem 1}
Define indicator random variables $I_i, i=1,...16$ by $I_i=1$ if a boy sits at seats i and his neighbor in the counterclockwise is a girl or a girl sits at seats i and her neighbor in the counterclockwise is a boy. Otherwise $I_i=0$. Then expected boy-girl neighbors are $E(X)=E(\sum_{i=1}^{16}I_i)=\sum_{i=1}^{16}E(I_i)=\sum_{i=1}^{16}(\frac{9}{16}*\frac{7}{15}+\frac{7}{16}*\frac{9}{15})=16*\frac{126}{240}=8.4$.

\section{Problem 2}
Independence implies uncorrelatedness. But the other way around is not necessarily true. For example, if $X$ is standard normal, then $X$ and $X^2$ are uncorreated, but not independent. $cov(X,X^2)=EXEX^2-EX^3=0$, so they are uncorrelated. But $P(0<X<1,X^>2)=0\neq P(0<X<1)P(X^2>1)$, so they are not independent. 

Reference: \url{https://en.wikipedia.org/wiki/Normally_distributed_and_uncorrelated_does_not_imply_independent}


\section{Problem 3}
%If you toss twice, after your first time toss, you continue if you get 1,2,3 since the expected value for one toss is $3.5$. So the payoff for tossing twice is $E(X)=1/2 * 3.5 + 1/6 * (4+5+6)= 4.25$. If you toss three times, after the first toss, you continue iff you get $1,2,3,4$. So the payoff for tossing three times is $4/6 * 4.25 + 1/6 * (5+6)=14/3$.  

Solve this problem backward. If one tosesed the 3rd time, his expectation is $(1+2+3+4+5+6)/6=3.5$. 
That means in the second toss if he gets 4,5,6 he should stop at 2nd toss. If he gets 1,2,3 he should use the 3rd chance.
So in the 2nd toss he has 1/6 chance to get 4,5,6 separately and 1/2 chance to get 3.5. The expectation is 4.25.
That means if in the 1st round he gets 5 or 6 he should stop otherwise continue.
In the 1st round he has 1/6 chance to get 5 or 6 seperately and 2/3 chance to get 4.25.
The expectation is $(5+6)/6+ 4.25*2/3 = 14/3=4.67$.

The value of the game is 4.67.

\section{Problem 4}
It is $\frac{1}{n}\sum_{i=1}^n\frac{1}{i}$. See 
$$http://math.stackexchange.com/questions/14190/average-length-of-the-longest-segment$$
If $V_n$ denotes the largest piece: then $P(V_n>x)=P(V_1>x\; {\rm or}\; V_2>x...)$. By inclusion and exclusion principal: 
$$P(V_n>x)=n(1-x)^{n-1}-{n \choose 2}(1-2x)^{n-1}+\cdot\cdot\cdot.$$

So $E(V_n)=\int_0^\infty P(V_n>x)dx=\sum_{k=1}^n{n \choose k}(-1)^{k-1}\int_0^{1/k}(1-kx)^{n-1}dx=\sum_{k=1}^n{n \choose k}(-1)^{k-1}\frac{1}{nk}=\frac{1}{n}\sum_{k=1}^n\frac{1}{k}.$

\section{Problem 5}
The determinat of the covariance matrix for $X,Y,Z$ is $det=1-r(r-0)+r(0-r)=1-2r^2$. Since $det\geq 0$, we get $0\leq 1-2r^2$. So $-\frac{\sqrt{2}}{2}\leq r\leq \frac{\sqrt{2}}{2}$. 

\section{Problem 6}
(a) Define stopping time by ${\rm min}\{n:S_n=-1\; {\rm or}\; 99\}$. Then since $S_N$ is a martingle, $E(S_N)=p_{-1}*(-1)+(1-p_{-1})*99=0$, which implies $p_{-1}=99/100$. SInce $S_N^2-N$ is a margintale, $E(N)=E(S_N^2)=\frac{99}{100}*(-1)^2+\frac{1}{100}*(99)^2= 99$. 

(b) If the drunk man hits the left locked door, he will go back, so it amount to the same problem as the (a) except here we need to replace $-1$ by $101$. So the expected number of steps he takes to go home is $E(N)=E(S_N^2)=\frac{99}{200}*101^2+\frac{101}{200}*99^2=101*99=9999$.


\section{Problem 7}
Ridge regression penalizes the size of the regression coefficients using $L^2$ norm.
Specifically, the ridge regression estimate $\hat{\beta}$ is defined as the value of $\beta$ that minimizes
$$\sum_i(y_i-x_i^T\beta)^2+\lambda\sum_i\beta_i^2.$$
The solution of the above ridge problem is $\hat{\beta}=(x^Tx+\lambda I)^{-1}x^Ty$.
Reference: \url{https://en.wikipedia.org/wiki/Least_squares}
Reference: \url{https://en.wikipedia.org/wiki/Tikhonov_regularization}


\section{Problem 8}
Lasso regression penalizes the size of the regression coefficients using $L^1$ norm.
Specifically, the lasso regression estimate $\hat{\beta}$ is defined as the value of $\beta$ that minimizes
$$\sum_i(y_i-x_i^T\beta)^2+\lambda\sum_i|\beta_i|.$$

Reference: \url{https://en.wikipedia.org/wiki/Least_squares}

\section{Problem 9}
Suppose that all the money in the world is less than $2^{46}$ US dollars. So after playing this game for 46 times, the money you could possibly get is always $2^{46}$ US dollars. Hence the expected payoff of this game is $46 \times 2^{46} * \sum_{n=47}^\infty \frac{1}{2^n}=46+1=47$. 


\section{Problem 10}
“default” means we want to use compiler-generated version of function, so don’t need to specify a body.
Here are cases we need default constructor:
When we want to force compiler to generate constructor.
When all parameters have default values;




\section{Problem 11}

\begin{framed}
\lstinputlisting{P11.cpp}
\end{framed}
Reference: \url{http://silviuardelean.ro/2012/06/05/few-singleton-approaches}


\section{Problem 12}

\begin{framed}
\lstinputlisting{quiz3Pro12.cpp}
\end{framed}


Reference: \url{https://leetcode.com/discuss/15153/a-clean-dp-solution-which-generalizes-to-k-transactions}



\section{Problem 13}

\begin{framed}
\lstinputlisting{quiz3Pro13.cpp}
\end{framed}


\section{Problem 14}
Yes. It’s sometimes (not always!) a great idea. For example, suppose all Shape objects have a common algorithm for printing, but this algorithm depends on their area and they all have a potentially different way to compute their area. In this case Shape’s area() method would necessarily have to be virtual (probably pure virtual) but Shape::print() could, if we were guaranteed no derived class wanted a different algorithm for printing, be a non-virtual defined in the base class Shape.

\begin{framed}
\lstinputlisting{P14.cpp}
\end{framed}
 
 


Reference: \url{https://isocpp.org/wiki/faq/strange-inheritance#calling-virtuals-from-base}


\section{Problem 15}

We can generate random numbers recurrencely use

\begin{eqnarray*}
	X_{n+1} = (aX_n+b)\mod m
\end{eqnarray*}
where a, b and m are large integers, and $X_{n+1}$is the next in $X$ as a series of pseudo-random numbers. The maximum number of numbers the formula can produce is the modulus, m.

Reference: \url{https://en.wikipedia.org/wiki/Random_number_generation}


\section{Problem 16}
We can flip fair coin n times to get a number below $2^n$. Then interpret this as a batch of m die flips, where m is the largest number so that $6^m<2^n$ (as already said, equality never holds here). If we get a number greater or equal to $6^m$ we must reject the value and repeat all n flips.


Reference: \url{http://cs.stackexchange.com/questions/29204/how-to-simulate-a-die-given-a-fair-coin}


\section{Problem 17}

\begin{framed}
\lstinputlisting{P17.cpp}
\end{framed}
 
 

Reference: \url{https://en.wikibooks.org/wiki/Puzzles/Logic_puzzles/Parachuted_Robots}


\end{document}



