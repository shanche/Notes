\documentclass[11pt]{article}

\usepackage[letterpaper,left=1.6in,right=1.6in,top=1.2in,bottom=1.2in]{geometry}

\usepackage[T1]{fontenc}
\usepackage[utf8]{inputenc}

\usepackage{lmodern}
\usepackage{amssymb}
\usepackage{amsmath}
\usepackage{amsthm}
\usepackage{mathtools}
\usepackage{hyperref}
\newtheorem{theorem}{Theorem}[section]
\theoremstyle{remark}
\newtheorem{remark}[theorem]{Remark}

\newcommand{\proof}{\bf Proof: \rm }%\nr}

\setcounter{secnumdepth}{0}
\usepackage{bbm}


%\usepackage[vlined]{algorithm2e}
\usepackage{algorithm,algorithmic}
\usepackage[]{algorithm2e}
\usepackage{paralist}
\usepackage{tikz}
\usepackage{xcolor,colortbl}
\usepackage{setspace}
\usepackage{listings}
\usepackage{color} %red, green, blue, yellow, cyan, magenta, black, white
\definecolor{mygreen}{RGB}{28,172,0} % color values Red, Green, Blue
\definecolor{mylilas}{RGB}{170,55,241}

\begin{document}
\setlength{\parindent}{0in}
\addtolength{\parskip}{0.1cm}
\setlength{\fboxrule}{.5mm}\setlength{\fboxsep}{1.2mm}
\newlength{\boxlength}\setlength{\boxlength}{\textwidth}
\addtolength{\boxlength}{-4mm}
\begin{center}\framebox{\parbox{\boxlength}{{\bf
QISHI QUIZ1 \hfill Due on 11:59PM SEP 29, 2015}\\
% TODO: fill in your own name, netID, and collaborators
Name: Bangrui Chen
}}
\end{center}
\vspace{1mm}

\definecolor{listinggray}{gray}{0.9}
\definecolor{lbcolor}{rgb}{0.9,0.9,0.9}
\lstset{
backgroundcolor=\color{lbcolor},
    tabsize=4,    
%   rulecolor=,
    language=[GNU]C++,
        basicstyle=\scriptsize,
        upquote=true,
        aboveskip={1.5\baselineskip},
        columns=fixed,
        showstringspaces=false,
        extendedchars=false,
        breaklines=true,
        prebreak = \raisebox{0ex}[0ex][0ex]{\ensuremath{\hookleftarrow}},
        frame=single,
        numbers=left,
        showtabs=false,
        showspaces=false,
        showstringspaces=false,
        identifierstyle=\ttfamily,
        keywordstyle=\color[rgb]{0,0,1},
        commentstyle=\color[rgb]{0.026,0.112,0.095},
        stringstyle=\color[rgb]{0.627,0.126,0.941},
        numberstyle=\color[rgb]{0.205, 0.142, 0.73},
%        \lstdefinestyle{C++}{language=C++,style=numbers}’.
}
\lstset{
    backgroundcolor=\color{lbcolor},
    tabsize=4,
  language=C++,
  captionpos=b,
  tabsize=3,
  frame=lines,
  numbers=left,
  numberstyle=\tiny,
  numbersep=5pt,
  breaklines=true,
  showstringspaces=false,
  basicstyle=\footnotesize,
%  identifierstyle=\color{magenta},
  keywordstyle=\color[rgb]{0,0,1},
  commentstyle=\color[rgb]{0,1,0},
  stringstyle=\color{red}
  }


% TODO: write your solution here
I didn't write down answers for all the problems since for some of them, my solution is the same with other people's.
\section{Math}
\subsection{Problem 1}
Solution: Denote the three points as point 1, point2 and point3. Consider for point1, the probability for point2 and point3 lies in the semi-circle that starts from point1 clockwise is $\frac{1}{2^{2}}$. Since point1, point2 and point3 are symmetric and we have three points in total, thus the final probability is $3\times \frac{1}{2^{2}}$. 

In general, if there are n points are the probability they all lie in a semi-circle is $n \times \frac{1}{2^{n-1}}$.

\subsection{Problem 7}
Without loss of generality, we assume the total length of N pieces is 1. An easy observation is that N pieces can form a polyhedron if and only if the total length of any $N-1$ pieces is greater than the length of the other one. Or equivalently, let we denote the length of the N pieces from the beginning to the end as $x_{1},\cdots,x_{N}$, then they can form a polyhedron if and only if
\begin{align}
\sum_{j\neq i}x_{j}\geq x_{i}>0 \nonumber \hspace{1mm}\forall i \\
\iff 0<x_{i}<\frac{1}{2} \nonumber \hspace{1mm}\forall i
\end{align}

Based on the above analysis, we know N pieces can't forma a polyhedron if and only if there is one piece $x_{i}\geq \frac{1}{2}$. Since
\begin{align}
P(\exists i,x_{i}\geq \frac{1}{2})&=\sum_{i=1}^{N}P(x_{i}\geq \frac{1}{2})\nonumber \\
&=NP(x_{1}\geq \frac{1}{2})\nonumber \\
&=\frac{N}{2^{N-1}} \nonumber
\end{align}
where the last equation is due to all the $x_{2},\cdots, x_{n}$ lies in $[\frac{1}{2},1]$.
Thus,
\begin{equation}
P(0<x_{i}<\frac{1}{2} \nonumber \hspace{1mm}\forall i)=1-P(\exists i,x_{i}\geq \frac{1}{2})=1-\frac{N}{2^{N-1}}. \nonumber 
\end{equation}
Thus, the probability that N pieces will form a polyhedron is $1-\frac{N}{2^{N-1}}$.

\subsection{Problem 8}
Here is a very interesting experiment:
\url{http://www.automated-trading-system.com/moving-median-better-indicator-than-moving-average/}.

The results of the experiments suggest that: the moving median will not increase robustness and the performance will also drop.


\section{Programming}
\subsection{Problem 12}
\begin{lstlisting}
class Solution {
public:
    ListNode* reverseBetween(ListNode* head, int m, int n) {
        ListNode* pointerM = head;
        if (m == 1) {
            ListNode* pointerN = head;
            for (int i = 1; i < n; i++) {
                pointerN = pointerN->next;
            }
            while (head != pointerN) {
                ListNode* temp = head;
                head = head->next;
                temp->next = pointerN->next;
                pointerN->next = temp;
            }
            return head;
        }
        for (int i = 1; i < m - 1; i++) {
            pointerM = pointerM->next;
        }
        ListNode* pointerN = pointerM;
        for (int i = 0; i <= n-m; i++) {
            pointerN = pointerN->next;
        }
        while (pointerM->next != pointerN) {
            ListNode* temp = pointerM ->next;
            pointerM->next = temp->next;
            temp->next = pointerN->next;
            pointerN->next = temp;
        }
        return head;
    }
};
\end{lstlisting}


\subsection{Problem 13}
\begin{itemize}
\item Solution 1: Use priority queue:
\begin{lstlisting}
// Solution 1:
bool ave_lower_P_1(int N, int M, float P, vector<float> price) {

  // imput the data into a priority queue
  priority_queue<float> price_new;
  for (int i = 0; i < N; i++) {
    price_new.push(-price[i]);
  }

  // calculate the average of the lowest M days closing price
  float sum = 0;
  for (int i = 0; i < M; i++) {
    sum += (-price_new.top());
    price_new.pop();
  }
  float ave = sum / M;

  cout << ave << endl;
  // return true or false
  return (ave <= P); 
}
\end{lstlisting}
\item Solution 2: Use sort in c++:
\begin{lstlisting}
// Solution 2:
bool ave_lower_P_2(int N, int M, float P, vector<float> price) {

  sort(price.begin(), price.end());

  // calculate the average of the lowest M days closing price
  float sum = 0;
  for (int i = 0; i < M; i++) {
    sum += (price[i]);
  }
  float ave = sum / M;

  cout << ave << endl;
  // return true or false
  return (ave <= P);
}
\end{lstlisting}
\item Solution 3: Use min heap in c++:
\begin{lstlisting}
// Solution 3:
bool ave_lower_P_3(int N, int M, float P, vector<float> price) {

  make_heap(price.begin(), price.end(), std::greater<int>());

  float sum = 0;
  for (int i = 0; i < M; i++) {
    sum += price.front();
    cout << price.front();
    pop_heap(price.begin(), price.end() - 1 - i, std::greater<int>());
    price.pop_back();
  }

  float ave = sum / M;

  cout << ave << endl;
  // return true or false
  return (ave <= P);
}
\end{lstlisting}

\end{itemize}


\subsection{Problem 17}
\begin{lstlisting}
class Solution {
public:
    int maxProfit(vector<int> &prices) {
        int profit = 0;
        for (int i = 1; i < prices.size(); i++) {
            if (prices[i] > prices[i-1]) {
                profit += prices[i] - prices[i - 1];
            }
        }
        return profit;
    }
};
\end{lstlisting}

\begin{appendix}
\section{Code for testing problem 12}
\begin{lstlisting}
#include <algorithm>
#include <assert.h>
#include <iostream>
#include <queue>
#include <vector>

using namespace std;


// Solution 1:
bool ave_lower_P_1(int N, int M, float P, vector<float> price) {

  // imput the data into a priority queue
  priority_queue<float> price_new;
  for (int i = 0; i < N; i++) {
    price_new.push(-price[i]);
  }

  // calculate the average of the lowest M days closing price
  float sum = 0;
  for (int i = 0; i < M; i++) {
    sum += (-price_new.top());
    price_new.pop();
  }
  float ave = sum / M;

  cout << ave << endl;
  // return true or false
  return (ave <= P);
}
// Solution 2:
bool ave_lower_P_2(int N, int M, float P, vector<float> price) {

  sort(price.begin(), price.end());

  // calculate the average of the lowest M days closing price
  float sum = 0;
  for (int i = 0; i < M; i++) {
    sum += (price[i]);
  }
  float ave = sum / M;

  cout << ave << endl;
  // return true or false
  return (ave <= P);
}

// Solution 3:
bool ave_lower_P_3(int N, int M, float P, vector<float> price) {

  make_heap(price.begin(), price.end(), std::greater<int>());

  float sum = 0;
  for (int i = 0; i < M; i++) {
    sum += price.front();
    cout << price.front();
    pop_heap(price.begin(), price.end() - 1 - i, std::greater<int>());
    price.pop_back();
  }

  float ave = sum / M;

  cout << ave << endl;
  // return true or false
  return (ave <= P);
}
int main() {
  // input the data
  int N,M;
  cout << "Input the number of trading days N" << endl;
  cin >> N;
  cout << "Input M (integer)" << endl;
  cin >> M;
  assert(M <= N && M >= 1);
  float P;
  cout << "Insert a price P" << endl;;
  cin >> P;
  assert( P > 0);
  cout << "Now insert the closing prices for the last " << N << " days" << endl;
  vector<float> price;
  float temp;
  for (int i = 0; i < N; i++) {
    cin >> temp;
    price.push_back(temp);
  }

  // check whether there exists M days average closing price less than P
  if (ave_lower_P_1(N,M,P,price)) {
      cout << "Yes, there exists " << M << " days" << endl;
  } else {
      cout << "No, there does not exist" << endl;
  }

  if (ave_lower_P_2(N,M,P,price)) {
      cout << "Yes, there exists " << M << " days" << endl;
  } else {
      cout << "No, there does not exist" << endl;
  }

  if (ave_lower_P_3(N,M,P,price)) {
      cout << "Yes, there exists " << M << " days" << endl;
  } else {
      cout << "No, there does not exist" << endl;
  }

  return 1;
}

\end{lstlisting}

\end{appendix}



\end{document}
