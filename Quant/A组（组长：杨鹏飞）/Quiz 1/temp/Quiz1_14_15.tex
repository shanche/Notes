%  Typeset with LaTeX format

\documentclass[a4paper, 11pt]{amsart}
\usepackage{amsmath, amssymb}
\usepackage{amsfonts}
\usepackage{mathrsfs}
\usepackage[arrow,matrix,curve,cmtip,ps]{xy}

\usepackage{amsthm}

%&biglatex
%\documentclass[11pt]{article}

\usepackage{graphicx}
\usepackage{bm}
\usepackage{color}
\usepackage{amsfonts}
\usepackage{amscd}
\usepackage{comment}
\usepackage{epsfig}
\usepackage{listings}
%\usepackage[all]{xy}

\usepackage[colorlinks=true, pdfstartview=FitV, linkcolor=red, citecolor=blue, urlcolor=green]{hyperref}
\usepackage{tikz,etex,bbm,xspace,hyperref}
\newcommand{\arxiv}[1]{\href{http://arxiv.org/abs/#1}{\tt arXiv:\nolinkurl{#1}}}


%Invert commenting for showkeys:
%\usepackage[notref]{showkey s}
\addtolength{\hoffset}{-1.5cm}
\addtolength{\textwidth}{3cm}
\allowdisplaybreaks



\newtheorem{theorem}{Theorem}[section]
\newtheorem{lemma}[theorem]{Lemma}
\newtheorem{proposition}[theorem]{Proposition}
\newtheorem{corollary}[theorem]{Corollary}
\newtheorem{notation}[theorem]{Notation}
\newtheorem*{theorem*}{Theorem}
\theoremstyle{remark}
\newtheorem{remark}[theorem]{Remark}
\newtheorem{definition}[theorem]{Definition}
\newtheorem{example}[theorem]{Example}
\newtheorem{question}{Question}
\newtheorem{claim}[theorem]{Claim}
\newtheorem*{solution*}{Solution}



%this has equations numbered within sections 1.1,1.2, ... 2.1,...
\numberwithin{equation}{section}


%-------------------------------------------
%       Begin Local Macros
%-------------------------------------------

\newcommand{\Z}{\mathbb{Z}}
\newcommand{\N}{\mathbb{N}}
\newcommand{\R}{\mathbb{R}}
\newcommand{\C}{\mathbb{C}}
\newcommand{\T}{\mathbb{T}}
\newcommand{\im}{\operatorname{im}}
\newcommand{\coker}{\operatorname{coker}}
\newcommand{\ind}{\operatorname{ind}}
\newcommand{\rank}{\operatorname{rank}}
\newcommand\mc[1]{\marginpar{\sloppy\protect\footnotesize #1}}
\newcommand{\carl}{\operatorname{Carl}}
\newcommand{\ci}[1]{_{ {}_{\scriptstyle #1}}}
%\ci --- Capital index
\newcommand{\ti}[1]{_{\scriptstyle \text{\rm #1}}}
%\ti --text index
\newcommand{\ut}[1]{^{\scriptstyle \text{\rm #1}}}
%\ut -- upper text index
\renewcommand{\labelenumi}{(\roman{enumi})}
\newcounter{vremennyj}
\newcommand\cond[1]{\setcounter{vremennyj}{\theenumi}\setcounter{enumi}{#1}\labelenumi\setcounter{enumi}{\thevremennyj}}
\renewcommand{\eqref}[1]{Equation (\ref{#1})}
\newcommand{\argmax}{\arg\!\max}
\newcommand{\argmin}{\arg\!\min}
         
%-------------------------------------------
%       End Local Macros
%-------------------------------------------





\begin{document}
\title[Code for Problem 14 and 15]{Code for Problem 14 and 15}

\maketitle






\section*{Problem 14}
\begin{lstlisting}
#include <iostream>
#include <limits>
#include <cmath>
using namespace std;

class expFunction {
private:
	double my_power(double x, int n);
public:
	expFunction(){};
	~expFunction(){};

	double my_exp(double x);
};

double expFunction::my_power(double x, int n) {
	if (n == 0) {
		return 1.0;
	}

	if (n%2 == 0) {
		return my_power(x, n/2) * my_power(x, n/2);
	} else {
		return x * my_power(x, n/2) * my_power(x, n/2);
	}
}

double expFunction::my_exp(double x)
{
	if (x < 0) {
		return 1.0/my_exp(-x);
	}

	// Round up x when x is large so that
	// e^x = 1 + x + ... + x^n/n! + O(x^n) converges faster.
	int roundup = ceil(x);
	double x_modified = x/roundup;

	double result = 1.0;
	double TaylorExpansionTerm = x_modified;
	int n = 1;
	while (TaylorExpansionTerm > numeric_limits<double>::min()) {
		result += TaylorExpansionTerm;
		TaylorExpansionTerm *= (x_modified/++n);
	}

	return my_power(result, roundup);
}


int main(int argc, char const *argv[]) {
	expFunction soln;

	double power;
	cout << "Input the power: " << endl;
	cin >> power;

	cout << "e^" << power << " = " << soln.my_exp(power) << endl;

	return 0;
}
\end{lstlisting}

\clearpage

\section*{Problem 15}
\begin{lstlisting}
#include <iostream>
#include <string>
#include <vector>
#include <ctime>
using namespace std;

class strMatch {
private:
	vector<int> next;
	void GetNext(const string& str);
public:
	strMatch(){};
	~strMatch(){};

	bool strStr(const string& haystack, const string& needle);
	bool strStrKMP(const string& haystack, const string& needle);
};

// Brute force: time O(m*n), space O(1)
bool strMatch::strStr(const string& haystack, const string& needle) {
	if (needle.empty()) {
		return true;
	}

	for (int i = 0; i < haystack.size(); i++) {
		if (haystack[i] == needle[0]) {
			bool match = true;
			for (int j = 0; j < needle.size(); j++) {
				if (haystack[i+j] != needle[j]) {
					match = false;
					break;
				}
			}

			if (match) {
				return true;
			}
		}
	}

	return false;
}

// KMP: time O(m + n), space O(n)
void strMatch::GetNext(const string& str) {  
    	next.push_back(-1);  
   	int i = -1;  
   	int j = 0;  
    	
   	while (j < str.size() - 1) {  
       		//str[i] - prefix,str[j] - suffix  
       		if (i == -1 || str[j] == str[i]) {  
           		i++;  
           		j++;  
 
           	 	if (str[j] != str[i]) {
            			next.push_back(i);    
            		} else {  
                		next.push_back(next[i]); 
            		} 

       		} else {  
           		i = next[i];  
       		}  
   	}

   	return;  
}

bool strMatch::strStrKMP(const string& haystack, const string& needle) {
	GetNext(needle);

	int i, j;
    	int haystackLen = haystack.size();
    	int needleLen = needle.size();

    	for (i = 0, j = 0; i < haystackLen && j < needleLen; ) {
		// currently, match!     
       		if (j == -1 || haystack[i] == needle[j]) {  
           		 i++;  
           		 j++;  
       		} else {  
           		 // currently, NOT match..    
            		 j = next[j];  
        	}  
    	}

	if (j == needle.size()) {
		return true;
	} else {
		return false;
	}  
}

int main(int argc, char const *argv[]) {
	strMatch soln;

	string haystack;
	cout << "Input haystack: ";
	getline(cin, haystack);

	string needle;
	cout << "Input needle: ";
	getline(cin, needle);

	clock_t now = clock();
	cout << "Brute force: " << soln.strStr(haystack, needle) << endl;
	clock_t after = clock();
	cout << "Brute force run-time: " << (after - now) / 
	(double)(CLOCKS_PER_SEC / 1000) << " ms" << endl;

	now = clock();
	cout << "KMP: " << soln.strStrKMP(haystack, needle) << endl;
	after = clock();
	cout << "KMP run-time: " << (after - now) / 
	(double)(CLOCKS_PER_SEC / 1000) << " ms" << endl;

	return 0;
}
\end{lstlisting}

\end{document}