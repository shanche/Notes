\documentclass[10pt, onecolumn, draftcls]{IEEEtran}
\usepackage{mathrsfs}
\usepackage{pdfpages}
\usepackage{graphicx}
\PassOptionsToPackage{draft}{graphicx}
\usepackage{amsmath}
\usepackage{epstopdf}
\usepackage{float}
\usepackage{amssymb} 
\usepackage{hyperref}
\usepackage{cite} 
\usepackage{caption}
\usepackage{subcaption}
\usepackage{bbm}
\usepackage{lipsum}
\usepackage{stfloats}
\usepackage{listings}
\usepackage{xcolor}
\lstset { %
    language=C++,
    backgroundcolor=\color{black!5}, % set backgroundcolor
    basicstyle=\footnotesize,% basic font setting
}


\newtheorem{thm}{Theorem}[section]
\newtheorem{cor}[thm]{Corollary}
\newtheorem{lem}[thm]{Lemma}
\newtheorem{rem}{Remark}
\newtheorem{Def}{Definition}

\DeclareMathOperator*{\esssup}{ess\,sup}
\DeclareMathOperator*{\essinf}{ess\,inf}

\begin{document}
\title{Solution 4}
\author{Group A}
\maketitle


\begin{enumerate}
\item Problem 3
Consider 7 girls $A, B, C, D, E, F, G$, starting from $A$, WLOG, say we have $AB$ and $AC$:
\begin{itemize}
\item Case I: if we have $BC$, then $D, E, F, G$ have to form a 4-cycle (e.g. $DE$, $EF$, $FG$, $GD$). In this case, we have in total $7!\times C_7^3\times 3 = 529200$.
\item Case II: if we do not have $BC$, again WLOG, say we have $BD$, let us consider
\begin{itemize}
\item Subcase (i): if we also have $CD$, then $A, B, C, D$ form a 4-cycle. This is already covered in Case I.
\item Subcase (ii): if not, again WLOG, say we have $CE$, then we cannot have $DE$ otherwise there is no arrangement satisfying the requirements. Up to this point,  it is easy to see that the only possibility would be $A, B, C, D, E, F, G$ form a 7-cycle.
\end{itemize}
In this case, we have in total $7!\times C_6^2\times 4\times 3\times 2 = 1814400$.
\end{itemize}
Enventually, we have $529200 + 1814400 = 2343600$ different possibilities.

\item Problem 8
\begin{align*}
\textup{success rate} = \frac{\textup{\# of successful shoot}}{\textup{\# of total shoot}} = \frac{S}{T}
\end{align*}
Let $S_0/T_0 < 0.5$ and for some $n\geq 0$, the first time we have $S_{n+1}/T_{n+1} > 0.5$. Consider $S_n$ and $T_n$, since $S_n/T_n \leq 0.5$, we must have $S_n = S_{n+1}-1$ and of course $T_n = T_{n+1} - 1$. Having this, we can estimate
\begin{align*}
S_n = S_{n+1} - 1 > \frac{T_{n+1}}{2} - 1 = \frac{T_n}{2} - \frac{1}{2}.
\end{align*}
Since both $S_n$ and $T_n$ are integers, we know
\begin{align*}
S_n \geq \left\lceil \frac{T_n}{2} \right\rceil\geq \frac{T_n}{2}.
\end{align*}
So, the only possibility is $S_n/T_n = 0.5$.


\item Problem 13
\begin{lstlisting}
#include <iostream>
#include <vector>
using namespace std;

class Solution{
    bool IsValid(vector<vector<char>>& broad, int x, int y){
        //  Check columes!
        for (int i=0; i<9; i++) {
            if ((i != x) && (broad[i][y]==broad[x][y])) {
                return false;
            }
        }
        
        //  Check rows!
        for (int j=0; j<9; j++) {
            if ((j != y) && (broad[x][j]==broad[x][y])) {
                return false;
            }
        }
        
        //  Check 3x3 box!
        for (int i=3*(x/3); i<3*(x/3+1); i++) {
            for (int j=3*(y/3); j<3*(y/3+1); j++) {
                if ((i != x) && (j != y) && (broad[i][j]==broad[x][y])) {
                    return false;
                }
            }
        }
        
        return true;
    }
public:
    Solution(){}
    ~ Solution(){}
    bool SudokuSolver(vector<vector<char>>& broad){
        for (int i=0; i<9; i++) {
            for (int j=0; j<9; j++) {
                //  Is the Sudoku still unsolved? Find a vacancy!!!
                if (broad[i][j]=='.') {
                    //  Try to fill in 1-9 into vacancy!!!
                    for (int k=0; k<9; k++) {
                        broad[i][j]='1'+k;
                        //  DFS!!! since we use SudokuSolver()!!!
                        if (IsValid(broad, i, j && SudokuSolver(broad))) {
                            return true;
                        }
                        broad[i][j]='.';
                    }
                }
                //  Cannot solve
                return false;
            }
        }
        //  If NO vacancy found=>already solved!!!
        return true;
    }
};


\end{lstlisting}
 
\end{enumerate}
\end{document}















