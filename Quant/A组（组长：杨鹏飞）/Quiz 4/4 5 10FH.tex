\documentclass[10pt, onecolumn, draftcls]{IEEEtran}
\usepackage{mathrsfs}
\usepackage{pdfpages}
\usepackage{graphicx}
\PassOptionsToPackage{draft}{graphicx}
\usepackage{amsmath}
\usepackage{epstopdf}
\usepackage{float}
\usepackage{amssymb} 
\usepackage{hyperref}
\usepackage{cite} 
\usepackage{caption}
\usepackage{subcaption}
\usepackage{bbm}
\usepackage{lipsum}
\usepackage{stfloats}
\usepackage{listings}
\usepackage{xcolor}
\lstset { %
    language=C++,
    backgroundcolor=\color{black!5}, % set backgroundcolor
    basicstyle=\footnotesize,% basic font setting
}


\newtheorem{thm}{Theorem}[section]
\newtheorem{cor}[thm]{Corollary}
\newtheorem{lem}[thm]{Lemma}
\newtheorem{rem}{Remark}
\newtheorem{Def}{Definition}

\DeclareMathOperator*{\esssup}{ess\,sup}
\DeclareMathOperator*{\essinf}{ess\,inf}

\begin{document}
\title{Solution 3}
\author{Group A}
\maketitle


\begin{enumerate}
\item 
Problem 4
...
\item
Problem 5
Only N=1 makes Wt^N a martingale.
Use Ito's Lemma, when N>=2, Wt^N has a non zero drift term, not martingale.
\item
Problem 10
Key Point: Converting to barycentric coordinates
point (p1.x, p1.y);
point (p2.x, p2.y);
point (p3.x, p3.y);
check whether (p.x,p.y) is in the triangle.

double alpha = ((p2.y - p3.y)*(p.x - p3.x) + (p3.x - p2.x)*(p.y - p3.y)) /
            ((p2.y - p3.y)*(p1.x - p3.x) + (p3.x - p2.x)*(p1.y - p3.y));
double beta = ((p3.y - p1.y)*(p.x - p3.x) + (p1.x - p3.x)*(p.y - p3.y)) /
            ((p2.y - p3.y)*(p1.x - p3.x) + (p3.x - p2.x)*(p1.y - p3.y));
double gamma = 1.0f - alpha - beta;

All alpha, beta and gamma large than zero, then the point is in the triangle.
\begin{lstlisting}
#Code Template
\end{lstlisting}
 
\end{enumerate}
\end{document}















