\documentclass[11pt]{article}

\usepackage[letterpaper,left=1.6in,right=1.6in,top=1.2in,bottom=1.2in]{geometry}

\usepackage[T1]{fontenc}
\usepackage[utf8]{inputenc}

\usepackage{lmodern}
\usepackage{amssymb}
\usepackage{amsmath}
\usepackage{amsthm}
\usepackage{mathtools}
\usepackage{hyperref}
\newtheorem{theorem}{Theorem}[section]
\theoremstyle{remark}
\newtheorem{remark}[theorem]{Remark}

\newcommand{\proof}{\bf Proof: \rm }%\nr}

\setcounter{secnumdepth}{0}
\usepackage{bbm}


%\usepackage[vlined]{algorithm2e}
\usepackage{algorithm,algorithmic}
\usepackage[]{algorithm2e}
\usepackage{paralist}
\usepackage{tikz}
\usepackage{xcolor,colortbl}
\usepackage{setspace}
\usepackage{listings}
\usepackage{color} %red, green, blue, yellow, cyan, magenta, black, white
\definecolor{mygreen}{RGB}{28,172,0} % color values Red, Green, Blue
\definecolor{mylilas}{RGB}{170,55,241}

\begin{document}
\setlength{\parindent}{0in}
\addtolength{\parskip}{0.1cm}
\setlength{\fboxrule}{.5mm}\setlength{\fboxsep}{1.2mm}
\newlength{\boxlength}\setlength{\boxlength}{\textwidth}
\addtolength{\boxlength}{-4mm}
\begin{center}\framebox{\parbox{\boxlength}{{\bf
QISHI QUIZ2 \hfill Due on 11:59PM OCT 12, 2015}\\
% TODO: fill in your own name, netID, and collaborators
Name: Bangrui Chen
}}
\end{center}
\vspace{1mm}

\definecolor{listinggray}{gray}{0.9}
\definecolor{lbcolor}{rgb}{0.9,0.9,0.9}
\lstset{
backgroundcolor=\color{lbcolor},
    tabsize=4,    
%   rulecolor=,
    language=[GNU]C++,
        basicstyle=\scriptsize,
        upquote=true,
        aboveskip={1.5\baselineskip},
        columns=fixed,
        showstringspaces=false,
        extendedchars=false,
        breaklines=true,
        prebreak = \raisebox{0ex}[0ex][0ex]{\ensuremath{\hookleftarrow}},
        frame=single,
        numbers=left,
        showtabs=false,
        showspaces=false,
        showstringspaces=false,
        identifierstyle=\ttfamily,
        keywordstyle=\color[rgb]{0,0,1},
        commentstyle=\color[rgb]{0.026,0.112,0.095},
        stringstyle=\color[rgb]{0.627,0.126,0.941},
        numberstyle=\color[rgb]{0.205, 0.142, 0.73},
%        \lstdefinestyle{C++}{language=C++,style=numbers}’.
}
\lstset{
    backgroundcolor=\color{lbcolor},
    tabsize=4,
  language=C++,
  captionpos=b,
  tabsize=3,
  frame=lines,
  numbers=left,
  numberstyle=\tiny,
  numbersep=5pt,
  breaklines=true,
  showstringspaces=false,
  basicstyle=\footnotesize,
%  identifierstyle=\color{magenta},
  keywordstyle=\color[rgb]{0,0,1},
  commentstyle=\color[rgb]{0,1,0},
  stringstyle=\color{red}
  }


% TODO: write your solution here

\section{Math}
\subsection{Problem 5}
Choose two random variables X and Y that uniformly lie in [0,1]. Let $A=\min(A,B)$ and $B=\max(A,B)$. Without loss of generality, we divide the stick into three pieces and let the length of the first piece be A, the second piece has length B-A and the third piece has length $1-B$. We first calculate the expected length for the smallest one. Suppose $S=\min(A,B-A,1-B)$, then the cdf of S is the following:
\begin{align}
F(a) &= 1 - P(A\geq a, B-A\geq a, 1-B\geq a) \nonumber \\
&=1-P(x\geq a, y\geq x, y-x \geq a, 1- y\geq a) - P(y\geq a, x\geq y, x-y\geq a, 1-x\geq a) \nonumber \\
&=1-2 P(x\geq a, y\geq x, y-x \geq a, 1- y\geq a) \nonumber \\
&=1-2 * P(x\geq a, y\geq x+a, y\leq 1-a) \nonumber \\
&=1-(1-3a)^{2}, \nonumber 
\end{align}
for $a\leq \frac{1}{3}$. Thus we know the pdf is 6(1-3a), 
\begin{align}
E[S]=\int_{0}^{\frac{1}{3}}6a(1-3a)da=\frac{1}{9}. \nonumber 
\end{align}
Similarly, if we denote the length of the largest one as $L=\max(A,B-A,1-B)$, then the cdf of L is the following:
\begin{align}
F(a) &= P(A\leq a, B-A\leq a, 1-B\leq a) \nonumber \\
&=P(x\leq a, y\geq x, y-x \leq  a, 1- y\leq a) - P(y\leq a, x\geq y, x-y\leq a, 1-x\leq a) \nonumber \\
&=2 P(x\leq a, y\geq x, y-x \leq a, 1- y\leq a) \nonumber \\
&=2 * P(x\leq a, 1-a \leq y\leq a+x, y\geq x) \nonumber \\
&=\begin{cases}
    -3a^{2}+6a-2      & a\geq \frac{1}{2}\\
    (3a-1)^{2}  & \frac{1}{3}\leq a\leq \frac{1}{2}\\
  \end{cases}. \nonumber 
\end{align}
Thus, we know the pdf is
\begin{align}
f(a)&=\begin{cases}
    6-6a      & a\geq \frac{1}{2}\\
    6(3a-1)  & \frac{1}{3}\leq a\leq \frac{1}{2}\\
  \end{cases}. \nonumber 
\end{align}

\begin{equation}
E[L]=\int_{\frac{1}{3}}^{\frac{1}{2}}6a(3a-1)da + \int_{\frac{1}{2}}^{1}6a-6a^{2}da = \frac{1}{9}+\frac{1}{2}=\frac{11}{18}.\nonumber 
\end{equation}
Since the all three pieces sum up to 1, we know the expected length of the middle size one is $1-\frac{11}{18}-\frac{1}{9}=\frac{1}{3}$.




If we divide the stick into n segments then the $k_{th}$ largest segment will have expected length $\frac{1}{n}(\frac{1}{n}+\cdots+\frac{1}{k})$\cite{os}\cite{stack}.

\subsection{Problem 7}

Based on Theorem 3.7 in \cite{es}, we know when a new variable is added to the original regression, the change in $R^{2}$ followed the following formula:
\begin{equation}
R_{1,2}^{2}=R_{1}^{2} + (1-R_{1}^{2})r^{2}, 
\end{equation}
where the explicit formula for $r^{2}$ can be found on page 44 in \cite{es}. Since$0\leq r^{2}\leq 1$, thus we know
\begin{equation}
0.2\leq R_{1,2}\leq 1.
\end{equation}





\section{Programming}
\subsection{Problem 11}
\begin{lstlisting}
/**
 * Definition for singly-linked list.
 * struct ListNode {
 *     int val;
 *     ListNode *next;
 *     ListNode(int x) : val(x), next(NULL) {}
 * };
 */
class Solution {
public:
    void deleteNode(ListNode* node) {
        while (node->next->next != NULL) {
            node->val = node->next->val;
            node = node->next;
        }
        node->val = node->next->val;
        node->next = NULL;
    }
};

\end{lstlisting}


\subsection{Problem 17}
\begin{lstlisting}
class Solution {
public:
    int maxProfit(vector<int> &prices) {
        if (prices.size() < 2) {
            return 0;
        }
        vector<int> forward;
        vector<int> backward;
        forward.push_back(0);
        backward.push_back(0);
        
        int valley = prices[0];
        for (int i = 1; i < prices.size(); i++) {
            forward.push_back(max(forward[forward.size() - 1], prices[i] - valley));
            valley = min(valley, prices[i]);
        }
        
        int top = prices[prices.size() - 1];
        for (int i = prices.size() - 2; i >= 0; i--) {
            backward.insert(backward.begin(), max(backward[0], top - prices[i]));
            top = max(top, prices[i]);
        }
        
        int profit = 0;
        for (int i = 0; i < prices.size(); i++) {
            profit = max(profit, forward[i] + backward[i]);
        }
        return profit;
    }
};
\end{lstlisting}

\begin{thebibliography}{9}

\bibitem{es}
H. G. William, \emph{\LaTeX: Econometric Analysis}, 7th Edition, 2010

\bibitem{os}
  H.A. David and H.N.Nagaraja
  \emph{\LaTeX: Order Statistics},
  Addison Wesley, Massachusetts,
  3rd edition,
  2003.
\bibitem{stack} \url{http://math.stackexchange.com/questions/13959/if-a-1-meter-rope-is-cut-at-two-uniformly-randomly-chosen-points-what-is-the-av}

\end{thebibliography}


\end{document}
