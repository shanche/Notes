\documentclass[10pt, onecolumn, draftcls]{IEEEtran}
\usepackage{mathrsfs}
\usepackage{pdfpages}
\usepackage{graphicx}
\PassOptionsToPackage{draft}{graphicx}
\usepackage{amsmath}
\usepackage{epstopdf}
\usepackage{float}
\usepackage{amssymb} 
\usepackage{hyperref}
\usepackage{cite} 
\usepackage{caption}
\usepackage{subcaption}
\usepackage{bbm}
\usepackage{lipsum}
\usepackage{stfloats}
\usepackage{listings}
\usepackage{xcolor}
\lstset { %
    language=C++,
    backgroundcolor=\color{black!5}, % set backgroundcolor
    basicstyle=\footnotesize,% basic font setting
}


\newtheorem{thm}{Theorem}[section]
\newtheorem{cor}[thm]{Corollary}
\newtheorem{lem}[thm]{Lemma}
\newtheorem{rem}{Remark}
\newtheorem{Def}{Definition}

\DeclareMathOperator*{\esssup}{ess\,sup}
\DeclareMathOperator*{\essinf}{ess\,inf}

\begin{document}
\title{Solution 3}
\author{Group A}
\maketitle


\begin{enumerate}
\item Problem 6

(a) Denote the drunk man's position at time t as $P_{t}$, then we know $P_{t}$ is a martingale and $P_{t}^{2}-t$ is also a martingale. Denote stopping time $\tau$ as the first time the drunk man at position -1 or 99. Then we know a martingale stops at a stopping time is also a martingale. Thus if we denote p as the probability that the drunk man will first visit -1 and q as the probability that the drunk man will first visit 99, we have
\begin{align}
p+q&=1\nonumber\\
p\times (-1) + q\times 99 &= 0.\nonumber 
\end{align}
Thus, we know $p=\frac{99}{100}$ and $q=\frac{1}{100}$. Since
\begin{equation}
E[P_{\tau}-\tau] = (-1)^{2}\times p + 99^{2}\times q - E[\tau]=0,\nonumber 
\end{equation}
thus we know $E[\tau]=99$.

(b) If the left door is locked, then when the drunk man hit -1, he still need to move right for 100 steps. Since left and right are symmetric, we know this is equivalent that we have two doors at -101 and 99. Thus, by the same analysis as part (a), we know $E[\tau]=9999$.

\item Problem 8

The LASSO, which is short for Least Absolute Shrinkage and Selection Operator, is a regression method that involves penalizing the absolute size of the regression coefficients. Its objective function is trying to minimize the following:
\begin{align}
\|Y-X\omega\|_{2}^{2} + \alpha\|\omega\|_{1}. \nonumber 
\end{align}
Compared to ordinary linear regression, the L1 penalization term can yield sparse models, and thus it can be used to do feature selection.

\item Problem 13

\begin{lstlisting}
class Solution {
public:
    int singleNumber(int A[], int n) {
        int result = 0;
        for (int i = 0; i < n; i++) {
            result = result ^ A[i];
        }
        return result;
    }
};\end{lstlisting}

\item Problem 14

The solution is found on this website: \url{https://isocpp.org/wiki/faq/strange-inheritance#calling-virtuals-from-base}.

Yes. It’s sometimes (not always!) a great idea. For example, suppose all Shape objects have a common algorithm for printing, but this algorithm depends on their area and they all have a potentially different way to compute their area. In this case Shape’s area() method would necessarily have to be virtual (probably pure virtual) but Shape::print() could, if we were guaranteed no derived class wanted a different algorithm for printing, be a non-virtual defined in the base class Shape.
\begin{lstlisting}
#include "Shape.h"
void Shape::print() const
{
    float a = this->area();  // area() is pure virtual
    // ...
}
\end{lstlisting}

\end{enumerate}

\end{document}















