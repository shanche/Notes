\documentclass[10pt, onecolumn, draftcls]{IEEEtran}
\usepackage{mathrsfs}
\usepackage{pdfpages}
\usepackage{graphicx}
\PassOptionsToPackage{draft}{graphicx}
\usepackage{amsmath}
\usepackage{epstopdf}
\usepackage{float}
\usepackage{amssymb} 
\usepackage{hyperref}
\usepackage{cite} 
\usepackage{caption}
\usepackage{subcaption}
\usepackage{bbm}
\usepackage{lipsum}
\usepackage{stfloats}
\usepackage{listings}
\usepackage{xcolor}
\lstset { %
    language=C++,
    backgroundcolor=\color{black!5}, % set backgroundcolor
    basicstyle=\footnotesize,% basic font setting
}


\newtheorem{thm}{Theorem}[section]
\newtheorem{cor}[thm]{Corollary}
\newtheorem{lem}[thm]{Lemma}
\newtheorem{rem}{Remark}
\newtheorem{Def}{Definition}

\DeclareMathOperator*{\esssup}{ess\,sup}
\DeclareMathOperator*{\essinf}{ess\,inf}

\begin{document}
\title{Solution 3}
\author{Group A}
\maketitle


\begin{enumerate}
\item \textsc{Problem 2.} In general, \emph{independence} implies \emph{uncorrelated}, but not vice versa.\\
E.g. Two random variables $X$ and $Y$ are independent if and only if $F_{X, Y}(x, y) = F_X(x)F_Y(y)$. They are uncorrelated if and if $\mathbb{E}[XY] = \mathbb{E}[X]\mathbb{E}[Y]$. Assuming independence, it is easy to check that
\begin{align*}
\mathbb{E}[XY] = \int xy dF_{X, Y}(x, y) = \int xdF_X(x)\cdot\int ydF_Y(y) = \mathbb{E}[X]\mathbb{E}[Y].
\end{align*}
Moreover, it can be shown that for any two measurable functions $f$ and $g$ where $\mathbb{E}[|f(X)|] < \infty$ and $\mathbb{E}[|g(Y)|] < \infty$, we have
\begin{align*}
\mathbb{E}[f(X)g(Y)] = \mathbb{E}[f(X)]\mathbb{E}[g(Y)].
\end{align*}
Now consider random variables $X = \textbf{1}_{[0, 1/4)} - \textbf{1}_{[1/4, 1/2)}$ and $Y = \textbf{1}_{[1/2, 3/4)} - \textbf{1}_{[3/4, 1)}$ on the probability space $\Omega = [0, 1)$ with standard Lebesgue measure. We have $\mathbb{E}[X] = 0$, $\mathbb{E}[Y] = 0$, and $\mathbb{E}[XY] = \mathbb{E}[0] = 0$, so they are uncorrelated. However, $\mathbb{E}[X^2] = 1/2$, $\mathbb{E}[Y^2] = 1/2$, but $\mathbb{E}[X^2Y^2] = \mathbb{E}[0] = 0$, so they are not independent.

\item \textsc{Problem 5.} Consider the correlation matrix of $X, Y, Z$, we have
\[ \left(
\begin{array}{ccc}
1 &r &0\\
r &1 &r\\
0 &r &1
\end{array}
\right) \]
This is a semi-definite matrix. So we need to have non-negative determinants of all its diagonal submatrics, i.e. $1 -r^2 \geq 0$ and $1-2r^2\geq 0$. Hence, we have 
$$-\frac{\sqrt{2}}{2}\leq r\leq \frac{\sqrt{2}}{2}.$$


\item \textsc{Problem 12.} This is a dynamic programming problem. Let 
\par global[i][j]: Max profit upto day i when we have completed at most j transactions
\par local[i][j]: Max profit upto day i when we have completed at most j transactions, \emph{AND} the last transaction is to sell the stock at day j\\
The state transition functions are\\
global[i][j] = max(local[i][j], global[i-1][j]) \\ 
local[i][j] = max(global[i-1][j-1] + max(diff, 0), local[i-1][j] + diff) 

\begin{lstlisting}
#include <iostream>
#include <vector>
#include <algorithm>
using namespace std;

class Solution {
    int helper(vector<int> &prices) {
        int profit = 0;
        for (int i = 0; i < prices.size() - 1; i++) {
            profit = max(profit, profit + prices[i + 1] - prices[i]);
        }
        return profit;
    }
public:
    int maxProfit(int k, vector<int> &prices) {
        int len = prices.size();
        if (len == 0) { return 0; }
        if (k >= len) { return helper(prices); }

        //  rolling array
        vector<int> max_local(k + 1, 0);
        vector<int> max_global(k + 1, 0);
        int diff;
        for (int i = 0; i < len - 1; i++) {
            diff = prices[i + 1] - prices[i];
            for (int j = k; j >= 1; j--) {
                max_local[j] = max(max_global[j - 1] + max(diff, 0), max_local[j] + diff);
                max_global[j] = max(max_local[j], max_global[j]);
            }
        }
        return max_global[k];
    }
};
\end{lstlisting}
 

\item \textsc{Problem 17.} (By Fei He) Idea: first, both robots move slowly to the right (right, left, right) until left robot hit 0 (right robot never hit zero), then left robot speed up (goright() only works for left robot) and finally they will meet

\begin{lstlisting}
while(!at_zero()){
    go_right();
    go_left();
    go_right();
} 
go_right();

\end{lstlisting}
\end{enumerate}
\end{document}















