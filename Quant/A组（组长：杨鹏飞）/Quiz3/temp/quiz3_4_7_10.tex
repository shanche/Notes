\documentclass{article}

\usepackage{amsmath}
\usepackage{amsfonts}
\usepackage{graphicx}
\usepackage{enumerate}
\usepackage{url}

\addtolength{\parskip}{\baselineskip} % add empty lines between paragraphs

\begin{document}
quiz3, 4+7+10

\section{Q4}
Ref: \url{http://math.stackexchange.com/questions/14190/average-length-of-the-longest-segment}

This problem is discussed extensively in David and Nagaraja's Order Statistics (pp. 133-135, and p. 153).

If $X_1,X_2,…,X_{n−1}$ denote the positions on the rope where the cuts are made, let $V_i=X_i−X_{i−1}$, where $X_0=0$ and $X_n=1$. So the $V_i$'s are the lengths of the pieces of rope.

The key idea is that the probability that any particular $k$ of the $V_i$'s simultaneously have lengths longer than $c_1,c_2,…,c_k$, respectively (where $\sum^k_{i=1}c_i \le 1$), is
$$(1−c_1−c_2− \ldots −c_k)^{n−1}.$$

This is proved formally in David and Nagaraja's Order Statistics, p. 135. Intuitively, the idea is that in order to have pieces of size at least $c_1,c_2,…,c_k$, all $n−1$ of the cuts have to occur in intervals of the rope of total length $1−c_1−c_2− \ldots −c_k$.

If $V_{(1)}$ denotes the shortest piece of rope, then for $x\le 1/n$,
$$P(V_{(1)}>x)=P(V_1>x,V_2>x,…,V_n>x)=(1−nx)^{n−1}.$$

Therefore,
$$E[V_{(1)}]=\int_0^{\infty} P(V_{(1)}>x)dx=\int^{1/n}_0(1−nx)^{n−1}=1/n^2.$$

David and Nagaraja also give the formula Yuval Filmus mentions (as Problem 6.4.2):

$$E[V_{(r)}]=1/n \sum_{j=1}^r 1/(n−j+1).$$

\section{Q7}
Ref: Element of Statistical learning

Ridge regression is one of the regularized  regression methods. Ridge regression estimate $\hat{\beta}$ by minimizing
$$
\sum_i (y_i-x_i^T\beta)^2 + \lambda \sum_{j=1}^p \beta_j^2
$$
where $\lambda$ is the tuning parameter.

The solution to the ridge regression is 
$$
\hat{\beta} = (X^TX + \lambda I)^(-1)X^Ty
$$

Applying the ridge regression penalty has the effect of shrinking the estimates towards zero, introducing bias but reducing the variance of the estimates

\section{Q10}
Ref: Thinking in C++ P340

A default constructor will only be automatically generated by the compiler if no other constructors are defined.  One case we want to define our own default constructor is,  for example, to declare a private constructor, and never define it, to prevent the compiler from implicitly defining any others.


\end{document}





